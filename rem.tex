%\documentclass[preprint]{sigproc}
\documentclass{acm_proc_article-sp}

% The following \documentclass options may be useful:
%
% 10pt          To set in 10-point type instead of 9-point.
% 11pt          To set in 11-point type instead of 9-point.
% authoryear    To obtain author/year citation style instead of numeric.

\setlength{\textfloatsep}{5pt}

\usepackage{flushend}
\usepackage{graphicx}
\usepackage{verbatim}
\usepackage{url}
\usepackage{alltt}
\renewcommand{\ttdefault}{txtt}

\usepackage{amsmath}
\usepackage{mathtools}
\everymath{\displaystyle}
\usepackage{xspace}

\newcommand{\CEU}{\textsc{C\'{e}u}\xspace}
\newcommand{\code}[1] {{\small{\texttt{#1}}}}
\newcommand{\DOFIN}{\code{do-finally}\xspace}
\newcommand{\FIN}{\code{finally}\xspace}

\newcommand{\ST}{\1\xrightarrow[~n~]{}\1}
\newcommand{\BT}{\xRightarrow[(i,E)]{}}
\newcommand{\LL}{\langle}
\newcommand{\RR}{\rangle}
\newcommand{\DS}{\displaystyle}


\newcommand{\1}{\;}
\newcommand{\2}{\;\;}
\newcommand{\3}{\;\;\;}
\newcommand{\5}{\;\;\;\;\;}
\newcommand{\ten}{\5\5}
\newcommand{\twenty}{\ten\ten}

\newenvironment{itemize*}%
  {\begin{itemize}%
    \setlength{\itemsep}{0pt}%
    \setlength{\parskip}{0pt}}%
  {\end{itemize}}

\usepackage{enumitem}
\setlist{nolistsep}

\usepackage{color}
\definecolor{light}{gray}{0.87}
\definecolor{dark}{gray}{0.30}
%\definecolor{light}{rgb}{.90,.90,.90}
\definecolor{darkgreen}{rgb}{0,.50,0}
\definecolor{darkblue}{rgb}{0,0,.50}
\definecolor{darkred}{rgb}{.50,0,0}
\definecolor{darkpur}{rgb}{.50,0,.50}

\usepackage{listings}
%\usepackage{textcomp}
\lstset{
columns=fullflexible,
%basicstyle=\ttfamily,
escapeinside={||},
mathescape=true,
    language=C, % choose the language of the code
    basicstyle=\fontfamily{pcr}\selectfont\footnotesize\color{black},
    keywordstyle=\color{black}\bfseries, % style for keywords
    numbers=none, % where to put the line-numbers
    numberstyle=\tiny, % the size of the fonts that are used for the line-numbers
    backgroundcolor=\color{light},
    showspaces=false, % show spaces adding particular underscores
    showstringspaces=false, % underline spaces within strings
    showtabs=false, % show tabs within strings adding particular underscores
    %frame=single, % adds a frame around the code
    tabsize=2, % sets default tabsize to 2 spaces
    %rulesepcolor=\color{gray}
    captionpos=t, % sets the caption-position to bottom
    breaklines=false, % sets automatic line breaking
    %breakatwhitespace=false,
    numbersep=2em,
    emph={par,or,xor,do,end,loop,await,emit,input,event,call,with,command,%
          var,and,then,else,C,return,pure,deterministic,nohold,finalize,%
          each, abort, when},
    emphstyle={\bfseries},
    commentstyle=\color{dark},
    %xleftmargin=20pt,
    %xrightmargin=20pt,
    framesep=20pt,
    %upquote=true,
    %aboveskip={1.5\baselineskip},
}

\begin{document}

% TODO: + controle
% TODO: externaly observable
% TODO: padrao de ir adicionando coisas em paralelo
% TODO: posso tirar isBlocked p/ cont e defer se cont(1)/defer(2)
%   e isBlocked levar em conta parametro(i) de cada comando
%   como faco na implementacao
% TODO: limitacao do finally, tem que esperar
% TODO: do-finally nem espera
% TODO: destacar caracteristica reativa p/ emit vs await
% TODO: inverter secoes?
% TODO:  Stack requer garantia de bounded para as cortinuacoes avançarem

\title {
    Reconciling Control and Dataflow Reactivity in Embedded Systems
}

\numberofauthors{1}
\author{
    \alignauthor
    Francisco Sant'Anna \hspace{1cm} Noemi Rodriguez \hspace{1cm} Roberto Ierusalimschy   \\
    \affaddr{Departamento de Inform\'atica --- PUC-Rio, Brasil} \\
    \email{\{fsantanna,noemi,roberto\}@inf.puc-rio.br}
}

\maketitle

\begin{abstract}
\CEU is a Esterel-based reactive language that prioritizes safety aspects for 
the development of reliable applications targeting highly constrained 
platforms.
%
Featuring a deterministic semantics, \CEU provides safe shared memory 
concurrency even when multiple lines of execution are active at the same time.
%
\CEU introduces a stack-based execution policy for internal events which 
enables advanced control mechanisms considering the context of embedded 
systems, such as exception handling.
%
The conjunction of internal events with shared-memory concurrency allows 
programs to express dependency among variables, which reconciles the control 
and dataflow reactive programming styles in a single language.
%As far as we know, \CEU is the first language to reconcile the control and 
%dataflow reactive styles.
%
%We present a formal description of \CEU and show how its synchronous and 
%static nature enables a compile-time analysis to ensure that reactions to the 
%environment are deterministic and execute with bounded memory and CPU time.
\end{abstract}

%\category{CR-number}{subcategory}{third-level}
\category{D.3.1}{Programming Languages}{Formal Definitions and Theory}
\category{D.3.3}{Programming Languages}{Language Constructs and Features}

\terms{Design, Languages, Reliability}

\keywords{Concurrency, Determinism, Embedded Systems, Safety, Static Analysis, 
Synchronous}

\section{Introduction}

Embedded systems are usually designed with safety and real-time requirements 
under constrained hardware platforms.
%
They are essentially reactive and interact permanently with the surrounding 
environment through input and output devices (e.g. buttons, timers, 
transceivers, etc.).

Software for embedded systems is usually developed in $C$, even though 
concurrent programming systems offering cooperative or preemptive 
multi-threading lack effective safety guarantees, being subject to unbounded 
execution~\cite{wsn.comparison}, race conditions and 
deadlocks~\cite{sync_async.threadsproblems}.

An established alternative to $C$ in the field of safety-critical embedded 
systems is the family of reactive synchronous languages \cite{rp.twelve}.
Two major styles of synchronous languages have evolved:
in the \emph{control}--\emph{imperative} style (e.g. Esterel 
\cite{esterel.design}), programs are structured with control flow primitives, 
such as parallelism, repetition, and preemption;
in the \emph{dataflow}--\emph{declarative} style (e.g. Lustre 
\cite{lustre.ieee91}), programs can be seen as graphs of values, in which a 
change to a value is propagated through its dependencies without explicit 
programming.

We believe that embedded systems programming can benefit from a new language 
that reconciles both reactive synchronous styles, while preserving typical $C$ 
features that programmers are familiarized, such as shared memory concurrency.
%
\CEU~\cite{ceu.sensys}
% \footnote{C\'eu is the Portuguese word for \emph{sky}.}
is a Esterel-based reactive programming language that provides a reliable yet 
powerful programming environment for embedded systems with some fundamental 
distinctions:
In this work we focus on the differences from \CEU to Esterel that introduce 
new programming functionalities:

%
\begin{itemize}
\item The execution order for memory operations rely on a deterministic 
semantics.
\item With a deterministic semantics, program executions are reproducible.
%
\item Internal events follow a stack-based execution policy.
\item The stack-based execution for internal events enables many advanced 
control mechanisms, such as exception handling, and dataflow programming.
% (like function calls in typical programming languages).
\end{itemize}

We present a formal semantics for the control primitives of \CEU and discuss an
alternative to Esterel's abortion primitive that is required for dataflow 
support.
This modification makes lines of executions to rejoin in an equivalent way to 
topological traversal in dependency graphs of dataflow 
languages~\cite{rp.twelve,frp.survey}.
%
We also show that, based on the stacked execution for internal events, \CEU can 
describe mutual data dependency without requiring an explicit \emph{delay} 
operator to break cycles~\cite{frtime.embedding,luagravity.sblp}.

%Based on the semantics, we also discuss the safety guarantees provided by 
%\CEU, such as deterministic behavior.

\CEU shares the same limitations with Esterel and synchronous languages in 
general:
computations that run in unbounded time (e.g., cryptography, image processing) 
do not fit the zero-delay hypothesis~\cite{rp.hypothesis}, and cannot be 
elegantly implemented.
%Also, both languages preclude the dynamic creation of lines of execution, as 
%they employ static analysis in order to provide safety warranties for 
%programs.

Nonetheless, previous work focusing on Wireless Sensor 
Networks~\cite{ceu.sensys} shows that the expressiveness of \CEU is sufficient 
for implementing a wide range of applications (e.g. network protocols and a 
radio driver), with a considerable reduction in code complexity and a small 
increase in resource usage in comparison to the 
state-of-the-art~\cite{wsn.nesc}.

The rest of the paper is organized as follows:
Section~\ref{sec.ceu} describes the main differences between \CEU and 
Esterel.
Section~\ref{sec.adv} shows how to implement some advanced control-flow 
mechanisms on top of \CEU internal events.
Section~\ref{sec.sem} shows how to implement some advanced control-flow 
mechanisms on top of \CEU internal events.
Section~\ref{sec.related} compares \CEU to existing synchronous and 
asynchronous languages for embedded systems.
Section~\ref{sec.conclusion} concludes the paper and makes final remarks.

\begin{comment}
%%%
In this work, we present \CEU, a reactive language targeting embedded systems 
that unifies both imperative and dataflow synchronous programming styles.
\CEU is based on a small set of reactive control primitives similar in 
functionality to Esterel's \cite{esterel.design}.
On top of this kernel, \CEU provides disciplined side effects, which together 
with internal events enable dataflow capabilities to the language.

Although, the first two items are similar, they are orthogonal.
One is at compile time, the other runtime.
Even without the first, the second is still valid.
The first warns about suspicious programs that still execute deterministically.

In our discussion, shared memory concerns not only variables, but also 
low-level accesses that ultimately use shared resources in the underlying 
platform (e.g., memory-mapped ports for I/O).

The stacked execution for internal events introduces support for a restricted 
form of subroutines that cannot express recursive definitions (either directly 
or indirectly), resulting in memory-bounded programs that preclude stack 
overflows.
\end{comment}

%\newpage
\section{Comparing C\'eu to Esterel}
\label{sec.ceu}

\CEU is a synchronous reactive language based on Esterel~\cite{esterel.ieee91} 
with support for multiple lines of execution known as \emph{trails}.
By reactive, we mean that programs are stimulated by the environment through 
input events that are broadcast to all awaiting trails.
By synchronous, we mean that any trail at any given time is either reacting to 
the current event or is awaiting another event;
in other words, trails are always synchronized at the current (and single) 
event.

Figure~\ref{lst.abro} shows side-by-side the implementations in Esterel and 
\CEU for the ``ABRO'' example with the following 
specification~\cite{esterel.primer}:
%
\emph{``Emit an output O as soon as two inputs A and B have occurred.
Reset this behavior each time the input R occurs''.}
%
The first line of the specification is almost identical in the two 
implementations, with a small syntactic mismatch between the `$\|$' operator 
and the \code{par/and} construct (lines 2-7 in Esterel, and 3-8 in \CEU).
%
The reset behavior of Esterel uses a \code{loop-each} syntactic sugar, which 
expands to the more primitive \code{abort-when} and serves as the same purpose 
of \CEU's \code{par/or} (to be discussed in Section~\ref{sec.ceu.abortion}).
In both cases, the occurrence of the event \code{R} aborts the awaiting 
statements and restarts the \code{loop}.

\begin{figure}[t]
\begin{minipage}[t]{0.45\linewidth}
\begin{lstlisting}[mathescape=true]
% ESTEREL
loop
  [
    await A
  $\|$
    await B
  ];
  emit O
each R
\end{lstlisting}
\end{minipage}
%
\begin{minipage}[t]{0.45\linewidth}
\begin{lstlisting}[numbers=left,xleftmargin=0.5em]
// CEU
loop do
    par/or do
        par/and do
            await A;
        with
            await B;
        end
        emit O;
    with
        await R;
    end
end
\end{lstlisting}
\end{minipage}
%\rule{8.4cm}{0.37pt}
\caption{ ``ABRO'' example in Esterel and \CEU. %\newline
{\small
%TODO.
}
\label{lst.abro}
}
\end{figure}

The languages have a strong imperative flavor, with explicit control flow 
through sequences, loops, and also assignments.
Being designed for control-intensive applications, they provide additional 
support for concurrent lines of execution and broadcast communication through 
events.
%
Esterel and \CEU also rely on a \emph{multiform notion of time}, in which 
programs advance in discrete and subsequent reactions to external 
\emph{signals} (\emph{events} in \CEU).
Internal computations within a reaction (e.g. expressions, assignments, and 
native calls) are considered to take no time in accordance with the synchronous 
(or \emph{zero-delay}) hypothesis~\cite{rp.hypothesis}.
The \code{await} statements are the only that halt a running reaction and allow 
a program to advance.

% TODO
% pause
% first-class timers
% fins
% integration w/ C

\subsection{External reactions and determinism}

In Esterel, an external reaction can carry simultaneous signals, while in \CEU, 
a single event defines a reaction.
%Figure~\ref{fig.reactions} illustrates this difference.
%TODO
%
The notion of time in Esterel is similar to that of digital circuits, in which 
multiple wires (signals) can be queried for its status (\emph{present} or 
\emph{absent}) on each clock tick.
%
\CEU more closely reflects event-driven programming, in which occurring events 
are sequentially and uninterruptedly handled by the program~\cite{EVT1,EVT2}.
%
Note that even with single-event rule there is still concurrency in \CEU given 
that multiple lines of execution may react to the same event.

Another difference between Esterel and \CEU is on their definition for 
determinism:
%
Esterel is deterministic with respect to reactive control, i.e., ``the same 
sequence of inputs always produces the same sequence of 
outputs''~\cite{esterel.primer}.
However, the order of execution for side-effect operations within a reaction is 
non-deterministic: ``if there is no control dependency and no signal 
dependency, as in ``\code{call P1() || call P2()}'', the order is unspecified 
and it would be an error to rely on it''~\cite{esterel.primer}.
%
In \CEU, when multiple trails are active at a time, as in
``\code{par/and~do~\_P1()~with~\_P2()~end}'', they are scheduled in the order 
they appear in the program text and run to completion (i.e. \code{\_P1} 
executes first)%
\footnote{
\CEU can call native functions by prefixing names with an underscore.
}.
%
This way, \CEU is deterministic also with respect to the order of execution of 
side effects within a reaction, which is formally described in 
Section~\ref{sec.sem}.

On the one hand, enforcing an execution order for concurrent operations may 
seen arbitrary and also precludes true parallelism.
On the other hand, it provides a priority scheme for trails (discussed in 
Section~\ref{sec.ceu.abortion}), and also ensures a reproducible execution for 
shared-memory programs.
%
For software development, we believe that determinism for side-effects is 
usually beneficial and is a design decision that makes sense for \CEU.
%For Esterel, however, is also used in hardware design~\cite{rp.twelve} where 
%parallelism is inherent.

\begin{comment}
%TODO
%\CEU has no \code{immediate} modifier
%no use for it
For instance, in Esterel, ``if a variable is written by some thread, then it 
can neither be read nor be written by concurrent 
threads''~\cite{esterel.primer}.

However, in Esterel, \emph{if a variable is written by some thread, then it can 
neither be read nor be written by concurrent threads}~\cite{esterel.primer} 
\end{comment}

\subsection{Thread abortion}
\label{sec.ceu.abortion}

The introductory example of Figure~\ref{lst.abro} illustrates how awaiting 
lines of execution can be aborted without being tweaked with synchronization 
primitives.
It is known that traditional multi-threaded languages cannot express thread 
termination safely~\cite{esterel.preemption,sync_async.threadsstop}.

The code fragments of Figure~\ref{lst.abortion} illustrate thread abortion in 
Esterel and \CEU.
Note that it is not clear in the examples if the calls should execute or not, 
given that the body and abortion event are the same.
For this reason, Esterel has the \emph{weak} and \emph{strong} variations for 
the \code{abort} statement.
In \CEU, given the scheduling rules, strong and weak abortions are determined 
by the order of trails within a \code{par/xor}, e.g., in the example \code{f1} 
executes and \code{f2} does not (weakly and strongly aborted, respectively).
%
The \code{par/or} of \CEU provides the alternative to execute both sides (if 
both terminate), which is fundamental to enable multiple dependencies from the 
same source in dataflow programming (to be discussed in
Sections~\ref{sec.sem}~and~\ref{sec.adv.frp}).
%
%Note that \code{\_f1} still executes before \code{\_f2} given the 
%deterministic scheduling of \CEU.

\begin{figure}[t]
\begin{minipage}[t]{0.35\linewidth}
\begin{lstlisting}
% ESTEREL
abort
    await S;
    call f1();
when S;
\end{lstlisting}
\end{minipage}
%
\begin{minipage}[t]{0.30\linewidth}
\begin{lstlisting}
// CEU (XOR)
par/xor do
    await S;
    _f1();
with
    await S;
    _f2();
end
\end{lstlisting}
\end{minipage}
%
\begin{minipage}[t]{0.30\linewidth}
\begin{lstlisting}
// CEU (OR)
par/or do
    await S;
    _f1();
with
    await S;
    _f2();
end
\end{lstlisting}
\end{minipage}
%\rule{8.4cm}{0.37pt}
\caption{ Thread abortion in Esterel and \CEU. %\newline
\label{lst.abortion}
}
\end{figure}
\begin{comment}
{\small
With strong abortion (default in Esterel), the call to \code{f1} is not 
executed (the \code{abort} can instead be prefixed with the \code{weak} 
keyword).
In \CEU, the \code{par/xor} provides strong and weak abortion depending on the 
chosen order for the trails.
The \code{par/or} provides XXX.
}
\end{comment}

%TODO:
%We are not confident with the defaults

% TODO
%delayed vs immediate abortion
%Practical experience has shown that delayed form were used much more often 
%than immediate ones.

\subsection{Internal events}
\label{sec.ceu.ints}

Esterel makes no semantic distinctions between internal and external 
signals~\cite{esterel.preemption}, both having only the notion of presence or
absence.
%
Another particularity of \CEU is how internal and external events behave 
differently:

\begin{itemize}
\item External events can be emitted only by the environment, while internal 
events only by the program.
\item A single external event can be active at a time, while multiple internal 
events can coexist within a reaction.
\item External events are handled in a queue, while internal events follow a 
stacked execution policy (like subroutine calls in typical programming 
languages).
\end{itemize}

Figure~\ref{lst.prints} illustrates the use of internal signals (events) in 
Esterel and \CEU.
%
For the Esterel version, there's no specific order the \code{printf} calls 
should execute.
%
For the \CEU version, on the occurrence of the event \code{START} the program 
behaves as follows (with the stack in emphasis):
%
{\small
\begin{enumerate}
\setlength{\itemsep}{0pt}
\item 1st trail awakes, emits \code{a}, and pauses;
    (\emph{stack: [1st]})
\item 2nd trail awakes, emits \code{b}, and pauses;
    (\emph{stack: [1st,2nd]})
\item 3rd trail awakes, prints $1$, and terminates;
    (\emph{stack: [1st,2nd]})
\item 2nd trail (on top of the stack) resumes, prints $2$, and terminates;
    (\emph{stack: [1st]})
\item 1st trail resumes, prints $3$, and terminates;
    (\emph{stack: []})
\item All trails have terminated, so the \code{par/and} rejoins, and the 
program also terminates;
\end{enumerate}
}

\begin{figure}[!t]
\begin{minipage}[t]{0.50\linewidth}
\begin{lstlisting}[mathescape=true]
% ESTEREL
input START;
signal A, B;
[[
    await START;
    emit A;
    call printf("3");
$\|$
    await A;
    emit B;
    call printf("2");
$\|$
    await B;
    call printf("1");
]]
\end{lstlisting}
\end{minipage}
%
\begin{minipage}[t]{0.45\linewidth}
\begin{lstlisting}
// CEU
input void START;
event void a, b;
par/and do
    await START;
    emit a;
    _printf("3");
with
    await a;
    emit b;
    _printf("2");
with
    await b;
    _printf("1");
end
\end{lstlisting}
\end{minipage}
%\rule{8.4cm}{0.37pt}
\caption{ Internal signals (events) in Esterel and \CEU. \newline
\label{lst.prints}
}
\end{figure}

\begin{comment}
{\small
The order of execution for the \code{printf} calls in Esterel is unspecified.
For \CEU, they always print $1$, $2$, and $3$, in this exact order.
}
Signals in Esterel have immediate feedback, i.e., an \code{await} statement can 
react to an \code{emit} in the same time it is reached.
In \CEU, only previously awaiting trails can react to an event.

immediate feedback
    flip-flops
- synthesis of digital circuits
immediate feedback (digital components as simple as flip-flops

causality complications
Note that both internal and external events are unbuffered, i.e., at the moment 
an event occurs, only previously awaiting trails can react to that instance.
\end{comment}

Internal events bring support for a limited form of subroutines, as illustrated 
int Figure~\ref{lst.sub}.
The subroutine \code{inc} is defined as a loop (lines 3-6) that continuously 
awaits its identifying event (line 4) and increments the value passed as 
reference (lines 5).
A trail in parallel invokes the subroutine in reaction to event \code{A} 
through the \code{emit} (lines 8-11).
Given the stacked execution for internal events, the calling trail pauses and 
the subroutine awakes.
Only after the subroutine ``returns'' (after the \code{await} in line 4), that 
the calling trail resumes and passes the assertion test.
 
\begin{figure}
\begin{lstlisting}[numbers=left,xleftmargin=2em]
event int* inc;    // function `inc'
par/or do
    loop do        // definitions are loops
        var int* p = await inc;
        *p = *p + 1;
    end
with
    var int v = 1;
    await A;
    emit inc => &v; // call `inc'
    _assert(v==2);  // after return
end
\end{lstlisting}
\caption{ The subroutine \code{inc} is defined in a loop, in parallel with the 
application.
\label{lst.sub}
}
\end{figure}

This form of subroutines has some significant limitations:

\begin{description}
\item[\emph{Single instance}:] Calls to a running subroutine have no effect.
For instance, if the subroutine in the example waits for another event before 
the loop, it cannot serve new requests.
%
\item[\emph{Single calling}:] Further calls to a subroutine in a reaction also 
have no effect.
To avoid unbounded execution, an \code{await} statement must be awaiting before 
a reaction starts.
%
\item[\emph{No recursion}:] Recursive calls to a subroutine also have no 
effect.
For the same reason of the \emph{single instance} property, a trail cannot be 
awaiting itself while running and the recursive call is ignored.
%
\item[\emph{No concurrency}:] If two trails in parallel try to call the same 
subroutine, only the first trail takes effect (based on deterministic 
scheduling).
The second call fails on the \emph{single calling} property.
\end{description}

\vspace{5pt}
\CEU provides no support for standard functions for a number of reasons:
\begin{itemize}
\item The interaction with other \CEU control primitives is not obvious (e.g., 
executing an $await$ or a $par/or$ inside a function).
\item They would still be restricted in some ways given the embedded context 
(e.g.  no recursion or closures).
\item Programs can always recur to $C$ for low-level operations.
%\item A dedicated primitive would behave just as described, being a matter of 
%syntactic sugar.
\end{itemize}

\begin{comment}
Regardless of the limitations, this form of subroutines is widely adopted in 
\CEU programs, given that they were designed to work with the other control 
mechanisms.
Keep in mind that the typical reactive organization of programs (awaiting an 
external stimulus, reacting to it, and going back to awaiting) does not demand 
unrestricted subroutines.
\end{comment}
In Section~\ref{sec.adv.frp}, we show that we can even take advantage of 
non-recursive subroutines to properly describe mutual dependency among trails 
in parallel.

%\newpage
\section{Advanced control mechanisms}
\label{sec.adv}

In this section, we explore the stacked execution for internal events in \CEU, 
demonstrating how it enables to derive support for \emph{exceptions} and 
\emph{dataflow programming} without requiring specific primitives.

\begin{comment}
Although the described mechanisms involve thoughtful techniques, they can be 
easily abstracted with compile-time macros taking advantage of the structured 
style of \CEU%
\footnote{Our programs in \CEU make extensive use of the \emph{m4} 
preprocessor.}.
As an exception, the \DOFIN construct to be presented in 
Section~\ref{sec.adv.fin} makes slight global additions to the program tree and 
requires a dedicated syntax.
\end{comment}

\subsection{Exception handling}
\label{sec.adv.excpt}

\begin{comment}
Exception handling can be provided by specialized programming language 
constructs (e.g., \code{try-catch} blocks in Java), but also with techniques 
using standard control-flow primitives (e.g., \code{setjmp/longjmp} in $C$).
\end{comment}

\CEU can naturally express different forms of exception handling on top of 
internal events without a specific construct.
%
As the illustrative example in Figure~\ref{lst.excpt1}, suppose an external 
entity periodically writes to a file and notifies the program through the event 
\code{ENTRY} carrying the number of available characters (line 2).
The normal flow is to wait \code{ENTRY} requests in a loop (lines 8-13) and 
request a \code{read} operation (line 11).
The low-level file operation \code{read} is defined as an internal event
working as a subroutine that fills the variable \code{buf} (shown later).
Relying on the stack-based execution for the \code{emit}, the code that 
manipulates the \code{buf} (line 12) is guaranteed to execute after it is 
filled.
Because this code does not handle failures, it is straight to the point and 
easy to follow.

\begin{figure}[t]
%\rule{8.5cm}{0.37pt}
\begin{lstlisting}[numbers=left,xleftmargin=2em]
// DECLARATIONS
input int ENTRY;        // new entry
var _FILE* f = <...>;   // reference to fie
var char[10] buf;       // current entry
event int read;         // reads into `buf'
event void excpt;       // callback for exceptions

// NORMAL FLOW
loop do
    var int n = await ENTRY;
    emit read => n;             // reads into `buf'
    _printf("line: %s\n", buf); // uses `buf'
end
\end{lstlisting}
\caption{ Normal flow to read file entries.
\label{lst.excpt1}
}
\end{figure}

Figure~\ref{lst.excpt2} defines the subroutine that performs the actual 
low-level \code{\_read} system call, which is placed in parallel with the 
normal flow.
The subroutine awaits requests in a loop (lines 5-10) and may emit
exceptions through event \code{excpt} on any error (lines 10-12).

\begin{figure}[t]
\begin{lstlisting}[numbers=left,xleftmargin=2em]
<...> // DECLARATIONS (as in previous code)
par/or do
    <...> // NORMAL FLOW (as in previous code)
with
    loop do     // READ subroutine
        var int n = await read;
        if (n > 10) || (_read(f,buf,n) != n) then
            emit excpt;  // read exception
        end
    end
end
\end{lstlisting}
\caption{ Low-level \code{read} operation is placed in parallel with the normal 
flow.
\label{lst.excpt2}
}
\end{figure}

\begin{figure}[t]
\begin{lstlisting}[numbers=left,xleftmargin=2em]
 <...> // DECLARATIONS
 par/or do
     par/xor do
         await excpt;           // catch exceptions
     with
         <...> // NORMAL FLOW
     end
 with
     <...> // READ (throw exceptions)
 end
\end{lstlisting}
\caption{ Exceptions are caught with a \code{par/xor} that strongly aborts the 
normal flow.
\label{lst.excpt3}
}
\end{figure}

Finally, to handle read exceptions, we use an additional \code{par/xor} in 
Figure~\ref{lst.excpt3} that \emph{strongly} aborts the normal flow on any 
exception.
Now, suppose the normal flow tries to read a string and fails.
The program behaves as follows (with the stack in emphasis):

{\small
\begin{enumerate}
\setlength{\itemsep}{0pt}
\item Normal flow invokes the read operation (line 11 of 
    Figure~\ref{lst.excpt1}) and pauses;\\
    \emph{stack: [norm]}
\item Read operation awakes (line 6 of Figure~\ref{lst.excpt2}), throws an 
    exception (line 8), and pauses;\\
    \emph{stack: [norm, read]}
\item Exception handler awakes (line 4 of Figure~\ref{lst.excpt3}) and 
terminates the \code{par/xor}, aborting the normal behavior and terminating the 
program. \\
    \emph{stack: []}
\end{enumerate}
}

This mechanism can also support resumption if the exception handler does not 
terminate its surrounding \code{par/xor}.
For instance, the new handler of Figure~\ref{lst.excpt4} catch exceptions in a 
loop and provides a default string to the normal flow.
The program behaves as follows (the two first steps are the same):

{\small
\begin{enumerate}[start=3]
\setlength{\itemsep}{0pt}
\item Exception handler awakes (line 4 of Figure~\ref{lst.excpt4}), assigns a 
default string to \code{buf} (line 5), and awaits the next exception (line 4).  \\
    \emph{stack: [norm, read]}
\item Read subroutine resumes (line 6 of Figure~\ref{lst.excpt2}), and awaits 
the next call. \\
    \emph{stack: [norm]}
\item Read call resumes (line 11 of Figure~\ref{lst.excpt1}), and uses 
\code{buf} normally (line 12), as if no exceptions had occurred. \\
    \emph{stack: []}
\end{enumerate}
}

\begin{figure}[t]
\begin{lstlisting}
 ...
    par/xor do
        loop do
            await excpt;  // catch exceptions
            buf = <...>;  // assigns a default
        end
    with
        <...> // NORMAL FLOW
    end
 ...
\end{lstlisting}
\caption{ Exception handling with resumption.
\label{lst.excpt4}
}
\end{figure}

Note that throughout the example, the normal flow of Figure~\ref{lst.excpt1} 
remained unchanged, with all machinery to handle exceptions placed around it.
%Also, although we use globals in the example (\code{f} and \code{buf}), 
%remember that they are guaranteed to be safely accessed.
With some syntactic sugar these exception mechanisms could be exposed in a 
higher level to developers.

TODO
%Esterel, no subroutines that affect the control
%also, abort is not resumable

\begin{comment}
In terms of memory usage, switching from the original normal flow (without 
exception throws) to the last example (with recovery) incurred extra 450 bytes 
of ROM and 24 bytes of RAM.

The presented approach for exceptions has the limitation that a file operation 
cannot be called twice within a reaction and that exception handlers cannot 
await other events, which are related to the single-call and single-instance 
property of subroutines in \CEU.
\end{comment}

\subsection{Dataflow programming}
\label{sec.adv.frp}

Reactive dataflow programming \cite{frp.survey} provides a declarative style to 
express dependency relationships among data.
%
Figure~\ref{} illustrates two concerns for correct
\emph{glitches} and \emph{cyclic dependencies}
In addit

Mutual dependency is a known issue in dataflow languages, requiring the 
explicit placement of a specific delay operator to avoid runtime
cycles~\cite{frtime.embedding,luagravity.sblp}.
This solution is somewhat \emph{ad hoc} and splits an internal dependency 
problem across two reactions to the environment.
%It also requires the mutual dependency to eventually converge to a value so 
%that variables do not affect each other forever.

b = s > s+1

loop do
var int s

loop do
    par/or do
        v1 = await s1;
    with
        v2 = await s1;
    end
    ;


\CEU can naturally express safe mutual dependencies, making it impossible to 
implement recursive definitions (as shown in Section~\ref{sec.ceu.ints}).
For instance, the program in Figure~\ref{lst.frp2} applies the temperature 
conversion formula between Celsius and Fahrenheit, so that whenever the value 
in one unit is set, the other is automatically recalculated (a problem proposed 
in~\cite{frp.survey}).

\begin{figure}[t]
%\rule{8.5cm}{0.37pt}
\begin{lstlisting}[numbers=left,xleftmargin=2em]
event int TC, TF;
var int tc, tf;
event int tc_evt, tf_evt;
par/or do
    loop do                // 1st trail
        tc = await tc_evt;
        emit tf_evt => (9 * tc / 5 + 32);
    end
with
    loop do                // 2nd trail
        tf = await tf_evt;
        emit tc_evt => (5 * (tf-32) / 9);
    end
with
    loop do
        var int v = await TC;   // 3rd trail
        emit tc_evt => v;
        <...>  // use `tc' or `tf'
    end
with
    loop do
        var int v = await TF;   // 4th trail
        emit tf_evt => v;
        <...>   // use `tc' or `tf'
    end
end
\end{lstlisting}
\caption{ A dataflow program with mutual dependency.
\label{lst.frp2}
}
\end{figure}

We first define the external events that signal changes, the variables to hold 
the temperatures, and corresponding internal events (lines 1-3).
Any change to a variable in the program must be signalled by an emit on the 
corresponding internal event so that dependent variables can react.
Then, we create two trails to await for internal changes and update the 
dependency relations among the temperatures (lines 5-8 and 10-13).
For instance, the first trail is a \code{loop} that waits for changes on 
\code{tc\_evt} (line 6) and signals the conversion formula to \code{tf\_evt} 
(line 7).
The behavior for the second trail that awaits \code{tf\_evt} (lines 10-13) is 
analogous.
The third and fourth trails (lines 15-19 and 21-25) await external updates in
loop to notify the internal changes;
The program behaves as follows (with the stack in emphasis):

{\small
\begin{enumerate}
\setlength{\itemsep}{0pt}
\item 1st and 2nd trail await \code{tc\_evt} and \code{tf\_evt};\\
    \emph{stack: []}
\item If \code{TC} occurs, 3rd trail signals a change to \code{tc\_evt} and 
    pauses;\\
    \emph{stack: [3rd]}
\item 1st trail awakes, sets \code{tc=0}, emits \code{tf\_evt}, and pauses;\\
    \emph{stack: [3rd,1st]}
\item 2nd trail awakes, sets \code{tf=32}, emits \code{tc\_evt}, and pauses;\\
    \emph{stack: [3rd,1st,2nd]}
\item no trails are awaiting \code{tc\_evt} (1st trail is paused), so 2nd trail 
    (on top of the stack) resumes, loops, and awaits \code{tf\_evt} again;\\
    \emph{stack: [3rd,1st]}
\item 1st trail resumes, loops, and awaits \code{tc\_evt} again;\\
    \emph{stack: [3rd]}
\item 3rd trail resumes \emph{with all dependencies resolved} and awaits the 
    next external change;\\
    \emph{stack: []}
\item ... (analogous behavior for further external occurrences)
\end{enumerate}
}

The complexity of the solution is disproportionate to the problem it solves, 
but illustrates the circular dependency issue (a similar example appears in 
other references~\cite{frp.survey,frtime.embedding}).
The bottom line is that dataflow techniques permit that complex dependency 
patterns are handled internally, providing well-defined entry points to 
application programmers (i.e. they would be required to write only the 3rd and 
4th trails in the example).

\begin{comment}
- same semantics: no additional complexity

- additional syntax

- efficient

- limitation: only static

%Altough xxx, no support for dynamic reconfiguration given the static nature of

By definition, a Lustre program may not contain syntac-
tically cyclic definitions. The commercial Scade tool pro-
vides an option for an even stronger constraint that forbids an
output of a node to be fed back as an input without an inter-
operator. Users generally accept these constraints.
vening
The first constraint ensures that there is a static dependence
order between flows and allows a very simple generation of
sequential code (the scheduling is just topological sorting).
%\cite{rp.twenty}

DEPRECATING ODERSKY
In order to prevent reactives from observing
inconsistent data (also called glitches) during and between
propagation cycles, we keep the dependency graph topolog-
ically sorted.

EMBEDDING
However, they enforce a topological order, which guarantees the absence of 
glitches and makes the state at the end of each update cycle well-defined.

ROY
This implementation can give a
glitch, for example if the new value of x reaches the addition before the new result of
the multiplication. This gives a temporary result of 17, which is incorrect. Glitches
are a source of nondeterminism that the implementation must avoid, for example by
compile-time preprocessing (doing a topological sort of operations) or thread scheduling
constraints. Some languages that implement this paradigm are Yampa (embedded in
Haskell) [27] and FrTime (embedded in Scheme) [12].

Most reactive programming languages eliminate glitches by arranging expressions 
in a topologically sorted graph [Cooper and Krishnamurthi 2006]; [Meyerovich
et al. 2009]; [Maier et al. 2010], thus ensuring that an expression is always evaluated
after all its dependents have been evaluated.
Most recent reactive implementations achieve glitch avoidance in reactive programs
running on a single computer, but not in distributed reactive programs. Avoiding
glitches in a distributing setting is not straightforward because of network failures,
delays and lack of a global clock. This is a potential sweet spot for future research
on distributed reactive systems that provide glitch freedom. We further discuss dis-
tributed reactive programming as an open issue in Section 5.
Also, an efficient reactive implementation should avoid unnecessary recomputations
of values that do not change. Dependent computations need not be recomputed if the
value they depend on is updated to a new value that is the same as the previous
value. Taking the same example above, suppose the value of var1 that is initially 1, is
afterwards updated to the same value (i.e., 1). In such a case, the values for var2 and
var3 need not to be recomputed as the value of var1 remained unchanged.
\end{comment}

%\newpage
%\section{Implementation of \CEU}
%\subsection{Operational semantics}
%\textbf{Formalization}
\section{The semantics of C\'eu}
\label{sec.sem}

We present a formal semantics of \CEU focusing on the control aspects of the 
language.
%
%\subsection{Abstract syntax}
%
The syntax for a subset of \CEU that is sufficient to describe all semantic 
peculiarities of the language is as follows:
%
%\begin{figure}[h]
%\rule{8.5cm}{0.37pt}
{\small
\begin{verbatim}
// primary statements    // description
nop(v)                   (constant value)
mem                      (any memory access)
await(e)                 (await event `e')
emit(e,v)                (emit event `e' passing `v')
break                    (loop escape)

// compound statements
mem ? p : q              (conditional)
p ; q                    (sequence)
loop p                   (repetition)
p and q                  (par/and)
p or q                   (par/or)
p xor q                  (par/xor)

// derived by semantic rules
awaiting(e,m)            (awaiting `e' since seqno `m')
emitting(t)              (emitting on stack level `t')
p @ loop p               (unwinded loop)
\end{verbatim}
}%
%\caption{
    %Reduced syntax of \CEU.
%\label{lst.syntax}
%}
%\end{figure}
%
A $nop$ represents a terminated computation associated with a constant value.
The $mem$ primitive represents all memory accesses, assignments, and $C$ 
function calls.
%
%As the challenging parts of \CEU reside on its control structures, we are not 
%concerned here with a precise semantics for side effects, but only with their 
%occurrences in programs.
%
The semantic rules generate three statements that the programmer cannot write:
the $awaiting$ avoids that an $await$ awakes during the same reaction it is 
reached;
the $emitting$ represents the continuation of an $emit$ and enables the desired 
stacked behavior for internal events;
a $loop$ is expanded with the special \code{`@'} separator (instead of 
\code{`;'}) to properly bind $break$ statements inside $p$ to the enclosing 
loop.

%\subsection{Operational semantics}

We use a \emph{small-step} structural operational semantics to formally 
describe a \emph{reaction chain} in \CEU, i.e., how a program behaves in 
reaction to a single external event.
% TODO
%\footnote{We could extend the semantics to describe the full execution of a 
%program by holding new incoming external events in a queue and processing them 
%in consecutive reaction chains that never overlap.}
%
The semantics require an explicit stack to properly nest the emission of 
internal events.
%
A complete reaction chain in \CEU is formalized as follows:
%
$$
\LL S, s, p \RR
    \xrightarrow[~~n~~]{~~*~~}
pop~\LL S', s', p' \RR
$$
%
\emph{At a given external sequence $n$, a program $p$ with a stack of events 
$S$ with top $s$ continuously progress until no transitions are possible.
After every transition, the top of the stack is popped if the program becomes 
blocked.}

At the beginning of each reaction, the external sequence number $n$ is 
incremented and the stack is initialized with the new external event, i.e., 
$s=1$ and $S=\{ 1 \mapsto ext \}$.
%
%The purpose of sequence number is to avoid that an $awaiting$ statement awakes 
%during the same reaction chain it is reached.
% TODO: why?
%
The top of the stack represents the single event being handled at a time:
it is initialized to the external event, but $emit$ statements can push new 
events on top of it.
When no transitions are possible in the current level, the stack is popped and 
the previous event is reactivated.
The reaction chain terminates when the level 1 is popped from the stack, 
signaling that the original external event has been completely handled.

To describe the full execution of a program with reaction chains in sequence, 
we need multiple ``invocations'' of the operational semantics, i.e.:
%
\begin{align*}
\LL \{ 1 \mapsto e1 \}, 1, p \RR
    & \xRightarrow[~~1~~]{}
\LL \{\}, 0, p' \RR
\\
\LL \{ 1 \mapsto e2 \}, 1, p' \RR
    & \xRightarrow[~~2~~]{}
\LL \{\}, 0, p'' \RR
\\
& ...
\end{align*}
%
Each invocation starts with the external event at the top of the stack and 
finishes with a modified program and an empty stack.
After each invocation, the sequence number is incremented.

Figure~\ref{fig.sem} shows the transitions rules for the complete semantics of 
\CEU.
At any time, at most one transition is possible.
This way, there is a single order of execution for side effects (represented by 
\code{mem} operations), and reaction chains are always deterministic.

\begin{figure}
%
{ \setlength{\jot}{7pt}
\begin{align*}
\LL S,s, mem \RR &\ST
\LL S,s, nop(v) \RR, v \in \mathbb{N}
    & \textbf{(mem)}        \\
%&~~(v~is~a~constant)
%%%
\LL S,s, await(e) \RR &\ST
\LL S,s, awaiting(e,n) \RR
    & \textbf{(await)}      \\
%%%
\LL S,s, awaiting(S(s),m) \RR &\ST
\LL S,s, nop \RR, m<n
    & \textbf{(awaiting)}   \\
%%%
\LL S,s, emit(e,v) \RR &\ST
\LL S \uplus \{(s+1) \mapsto e\}, \\
&~~~~~~~~~~s+1, emitting(s) \RR
    & \textbf{(emit)}       \\
%%%
\LL S,s, emitting(s) \RR &\ST
\LL S,s, nop \RR
    & \textbf{(emitting)}
\end{align*}
%}
%
%{ \setlength{\jot}{7pt}
\begin{eqnarray*}
& \frac
    {\DS m \ST m' }
% -----------------------------------------------------------
    {\DS \LL S,s, (m~?~p~:~q) \RR \ST
         \LL S,s, (m'~?~p~:~q)\RR }
    & \textbf{(if-adv)}       \\
%%%
& \LL S,s, (nop(v)~?~p~:~q) \RR \ST
  \LL S,s,  p \RR \1,\3 (v \neq 0)
    & \textbf{(if-true)}       \\
%%%
& \LL S,s, (nop(0)~?~p~:~q) \RR \ST
  \LL S,s, q \RR
    & \textbf{(if-false)}       \\
%%%
& \frac
    {\DS p \ST p' }
%   -----------------------------------------------------------
    {\DS \LL S,s, (p~;~q) \RR \ST \LL S,s, (p'~;~q) \RR }
    & \textbf{(seq-adv)}      \\
%%%
& \LL S,s, (nop~;~q) \RR \ST  \LL S,s, q \RR
    & \textbf{(seq-nop)}      \\
%%%
& \LL S,s, (break~;~q) \RR \ST \LL S,s, break \RR
    & \textbf{(seq-brk)}      \\
%\end{eqnarray*}
%}
%
%{ \setlength{\jot}{7pt}
%\begin{eqnarray*}
& \LL S,s, (loop~p) \RR \ST \LL S,s, (p~@~loop~p) \RR
    & \textbf{(loop-expd)}       \\
%%%
& \frac
    {\DS p \ST p' }
% -----------------------------------------------------------
    {\DS \LL S,s, (p~@~loop~q) \RR \ST \LL S,s, (p'~@~loop~q) \RR }
    & \textbf{(loop-adv)}    \\
%%%
& \LL S,s, (nop~@~loop~p) \RR \ST \LL S,s, loop~p \RR
    & \textbf{(loop-nop)}    \\
%%%
& \LL S,s, (break~@~loop~p) \RR \ST \LL S,s, nop \RR
    & \textbf{(loop-brk)}       \\
%\end{eqnarray*}
%}
%
%{ \setlength{\jot}{7pt}
%\begin{eqnarray*}
& \frac
    {\DS p \ST p' }
%   -----------------------------------------------------------
    {\DS \LL S,s, (p~par~q) \RR \ST \LL S,s, (p'~par~q) \RR }
    & \textbf{(par-adv1)}      \\
%%%
& \frac
    {\DS isBlocked(p) \1\wedge\1 q \ST q' }
%   -----------------------------------------------------------
    {\DS \LL S,s, (p~par~q) \RR \ST \LL S,s, (p~par~q') \RR }
    & \textbf{(par-adv2)}      \\
%%%
& \LL S,s, (break~par~q) \RR \ST \LL S,s, break \RR
    & \textbf{(par-brk1)}   \\
%%%
& \frac
    {\DS isBlocked(p) }
%   -----------------------------------------------------------
    {\DS \LL S,s, (p~par~break) \RR \ST \LL S,s, break \RR }
    & \textbf{(par-brk2)}       \\
%%%
& \LL S,s, (nop~and~q) \RR \ST \LL S,s, q \RR
    & \textbf{(and-nop1)}   \\
%%%
& \LL S,s, (p~and~nop) \RR \ST \LL S,s, p \RR
    & \textbf{(and-nop2)}   \\
%\end{eqnarray*}
%}
%
%{ \setlength{\jot}{7pt}
%\begin{eqnarray*}
%
%%%
& \frac
    {\DS isBlocked(q) \1\vee\1 q=nop}
%   -----------------------------------------------------------
    {\DS \LL S,s, (nop~or~q) \RR \ST \LL S,s, nop \RR }
    & \textbf{(or-nop1)}    \\
%%%
& \frac
    {\DS isBlocked(p)}
%   -----------------------------------------------------------
    {\DS \LL S,s, (p~or~nop) \RR \ST \LL S,s, nop \RR }
    & \textbf{(or-nop2)}    \\
%%%
& \LL S,s, (nop~xor~q) \RR \ST \LL S,s, nop \RR
    & \textbf{(xor-nop1)}   \\
%%%
& \frac
    {\DS isBlocked(p) }
%   -----------------------------------------------------------
    {\DS \LL S,s, (p~xor~nop) \RR \ST \LL S,s, nop \RR }
    & \textbf{(xor-nop2)}
%
\end{eqnarray*}
}
%
\caption{ The semantics of \CEU.
\label{fig.sem}
}
\end{figure}

A $mem$ reduces to a $nop$ in a single step (rule \textbf{mem}), representing 
the (bounded) execution of a side-effect operation.
Although we opted not to formalize side effects, the actual values yielded by 
$mem$ operations are required in conditionals (e.g., the value of a variable or 
the result of a $C$ function call).
(In the rules that an yielded $nop$ cannot appear in a conditional, we omit its 
generated value, i.e., we use $nop$ instead of $nop(v)$.)

An $await$ is simply transformed into an $awaiting$ that remembers the current 
external sequence number $n$ (rule \textbf{await}).
An $awaiting$ can only transit to a $nop$ (rule \textbf{awaiting}) if its 
referred event matches the top of the stack ($S(s)$) and its sequence number is 
smaller than the current one ($m<n$).
%
%Remember that in \CEU, the \code{await} statement returns the value associated 
%with the corresponding event: the yielded $mem$ represents the operation to 
%query that value.
%
An $emit$ transits to an $emitting$ and stacks its referred event (rule 
\textbf{emit}).
With the new stack level $s+1$, the $emitting(s)$ cannot transit, as rule 
\textbf{emitting} expects its parameter to match the current stack level.
This trick enforces the desired stack-based semantics for internal events.

The function \textbf{pop} (called after every transition) removes the top of 
the stack when the program becomes blocked, allowing that previously stacked 
$emitting$ operations progress:
%
\[
pop~\LL S, s, p \RR~ =
\begin{cases}
    \LL S, s-1, p \RR, &        isBlocked(p) \1\wedge\1 s>0     \\
    \LL S, s,   p \RR, & \neg ( isBlocked(p) \1\wedge\1 s>0 )
\end{cases}
\]
%
The recursive predicate $isBlocked$ is true only if all branches in parallel 
are hanged in $awaiting$ or $emitting$ operations that cannot transit:
%
{\small
\begin{align*}
  isBlocked(n,S,s, awaiting(e,m)) &= (e \neq S(s) \1\vee\1 m = n)   \\
  isBlocked(n,S,s, emitting(t))   &= (t \neq s)                     \\
  isBlocked(n,S,s, (p~;~q))       &= isBlocked(n,S,s,p)             \\
  isBlocked(n,S,s, (p~@~loop~q))  &= isBlocked(n,S,s,p)             \\
  isBlocked(n,S,s, (p~and~q))     &= isBlocked(n,S,s,p)~\wedge \\
                                     &~~~~isBlocked(n,S,s,q)             \\
  isBlocked(n,S,s, (p~or~q))      &= isBlocked(n,S,s,p)~\wedge \\
                                     &~~~~isBlocked(n,S,s,q)             \\
  isBlocked(n,S,s, *)             &= false \2  (nop,mem,await,      \\
                                  &    \5\5\5\2 emit,break,if,loop)   %\\
\end{align*}
}%
%\caption{ The recursive predicate $isBlocked$: only the \emph{awaiting} and 
%\emph{stacked} primitives can block and avoid transitions.
%\label{fig:isBlocked}
%
\begin{comment}
The first two cases query the $awaiting$ and $emitting$ operations to check if 
they are blocked.
Note that the test predicates are reversed in comparison to the semantic rules 
\textbf{awaiting} and \textbf{emitting} (i.e. they test when \emph{not} to 
advance).
Note also that $nop$ is not considered a blocked operation.

An emitted event remains at the top of the stack while there are matching 
$awaiting$ operations---only after the $isBlocked$ predicate becomes true that 
the $pop$ function takes effect.
This behavior provides the desired broadcast communication for events.
\end{comment}

The rules for conditionals and sequences are straightforward (\textbf{if-$*$} 
and \textbf{seq-$*$}).
%
Given that the semantics focus on control, note that rules \textbf{if-true} and 
\textbf{if-false} are the only to query $nop(v)$ values.
%For all other rules, we omit these values (e.g., \textbf{seq-nop}).

The rules for loops are analogous to sequences, but use \code{`@'} as 
separators to properly bind a break to its enclosing loop.
%
When a program first encounters a $loop$, it first expands its body in sequence 
with itself (rule \textbf{loop-expd}).
Rules \textbf{loop-adv} and \textbf{loop-nop} are similar to rules 
\textbf{seq-adv} and \textbf{seq-nop}, advancing the loop until it reaches a 
$nop$.
However, what follows the loop is the loop itself (rule \textbf{loop-nop}).
Note that if we used \code{`;'} as a separator in loops, rules 
\textbf{loop-brk} and \textbf{seq-brk} would conflict.
%
Rule \textbf{loop-brk} escapes the enclosing loop, transforming everything into 
a $nop$.

The rules with the \textbf{par} prefix are valid for all three parallel 
compositions $and$, $or$, $xor$ (substituting $par$ in the rules for each of 
them).
The rules force transitions on the left branch $p$ to occur before transitions 
on the right branch $q$ (rules \textbf{par-adv1} and \textbf{par-adv2}).
%
The deterministic behavior of the semantics relies on the \emph{isBlocked} 
predicate used in rule \textbf{par-adv2}, which requires the left branch $p$ to 
be blocked in order to allow the right transition from $q$ to $q'$.
%
The rules \textbf{par-brk1} and \textbf{par-brk2} deal with a $break$ in each 
of the parallel sides, which terminates the whole composition to escape the 
innermost loop (strongly aborting the other side).

The difference among the three parallel compositions consists in how to deal 
with one of the sides terminating with a $nop$.
%
For an $and$ composition, if one of the sides terminate, the composition is 
simply substituted by the other side (rules \textbf{and-nop1} and \textbf{and-nop2}).
%
For a parallel $or$, reaching a $nop$ in either of the sides should terminate 
the composition (rules \textbf{or-nop1} and \textbf{or-nop2}).
However, the other side must be blocked before being aborted.
%
In constrast, a $xor$ can strongly abort the trail in the right as soon as the 
left trail reaches a $nop$ (rule \textbf{xor-nop1}.
The rules \textbf{or-nop2} and \textbf{xor-nop2} are equivalent.

%\newpage
\section{Related work}
\label{sec.related}

\begin{comment}
\CEU is not the first Esterel-based language to employ deterministic 
scheduling~\cite{wsn.sol}

\CEU is strongly influenced by Esterel~\cite{esterel.ieee91}, but they are 
different in the fundamental aspect of dealing with events (signals in 
Esterel).
For instance, the stacked execution for internal events employed by \CEU 
greatly improves the expressiveness of the language as shown in 
Section~\ref{sec.adv}.
%For instance, dataflow programming is not feasible in Esterel, what resulted 
%in dedicated languages for that purpose~\cite{rp.twelve}.

Furthermore, Esterel is commonly used in hardware design, and its notion of 
time is similar to that of digital circuits, where multiple signals can be 
active at a clock tick.
In \CEU, instead of clock ticks, the occurrences of external events that define 
time units.
We believe that for software design, this approach simplifies the reasoning 
about concurrency.
For instance, the uniqueness of external events in \CEU is a prerequisite for 
its static analysis that enables safe shared-memory concurrency.
However, in Esterel, \emph{if a variable is written by some thread, then it can 
neither be read nor be written by concurrent threads}~\cite{esterel.primer} 
(this statement regards to the program text, not to a reaction chain).

More recently, Wireless Sensor Networks (WSNs) emerged as an active research 
area for highly constrained embedded concurrency, resulting in the development 
of many synchronous languages~\cite{wsn.protothreads,wsn.sol,wsn.osm}.

Protothreads \cite{wsn.protothreads} offer lightweight cooperative 
multithreading for embedded systems.
Its stackless implementation reduces memory consumption but precludes support 
for local variables.
\CEU also avoids the use of stacks for trails, but preserves support for locals 
by calculating the required memory at compile time.
%, as shown in Section~\ref{sec.safety.mem}.

SOL~\cite{wsn.sol} and OSM~\cite{wsn.osm} provide parallel state machines for 
WSNs, offering a formal and mature model for programming embedded systems.
However, the main contributions of \CEU, stacked execution for internal 
events and safe support for shared-memory concurrency, do not directly adapt to 
the state-machine formalism.

In common among the referred works is the agreement in providing low-level 
access (e.g., systems calls and shared-memory) and lock-free concurrency that 
precludes race conditions on programs.
However, they do not specify an execution order for tasks reacting to the same 
external stimulus~\cite{esterel.primer,wsn.protothreads,wsn.osm}.
This way, if two tasks access the same resource concurrently, even if the 
accesses are race free, the final outcome is nondeterministic.
As discussed in Section~\ref{sec.safety.det}, \CEU refuses programs with such 
behavior.

%{\small
%\begin{itemize}
%\item \emph{If there is no control dependency and no signal dependency, as in 
%\code{(call~P1()~||~call~P2())}, the order is unspecified and it would be an
%error to rely on it.}~\cite{esterel.primer}
%\item \emph{The protothreads mechanism does not specify any specific method to 
%invoke or schedule a protothread, this is defined by the system using 
%protothreads.}~\cite{wsn.protothreads}
%\item \emph{A single write access will always completely execute before the 
%next write access can occur. However, the order in which write accesses are 
%executed is arbitrary.}~\cite{wsn.osm}
%\item \emph{The parallel operator executes all its threads in a round-robin 
%manner according to the order of their declaration in the 
%program.}~\cite{wsn.sol}
%\end{itemize}
%}

%Regarding the last policy, we believe that our proposed static analysis is an 
%improvement over deterministic schedulers.

On the opposite side of the spectrum of concurrency models, asynchronous 
languages for embedded systems~\cite{wsn.mantisos,arduino.occam}
assume time independence among processes and are more appropriate for 
applications with a low synchronization rate or for those involving
algorithmic-intensive problems.

%\begin{commen}
% TODO
SHIM is an asynchronous language that enforces synchronous communications among 
processes, providing a deterministic execution model.
SHIM distinguishes from typical asynchronous languages given that
The use of point-to-point communication, typical in CSP-like 
languages~\cite{async.csp}, leads to a different programming mindset.
No shared-memory
no hierarchies (e.g., \code{par/or} compositions)
%\end{commen}

%TODO: limitation static (mantis, proto, occam tb!)

% TODO: leds, buffer overflows

Asynchronous models are also employed in real-time operating systems to provide 
response predictability, typically through prioritized 
schedulers~\cite{wsn.mantisos,wsn.oses,freertos}.
Even though \CEU ensures bounded execution for reactions, it cannot provide 
hard real-time warranties.
For instance, assigning different priorities for trails would break lock-free 
concurrency (i.e., breaking correctness is worse than breaking timeliness).
%synchronous model and
%the static analysis, which are required for

%That said, some embedded systems do require prioritized scheduling to meet 
%deadlines for critical tasks, even if it involves extra complexity to deal 
%%with synchronization issues.
Fortunately, \CEU and RTOSes are not mutually exclusive, and we can foresee a 
scenario in which multiple \CEU programs run in different RTOS threads and 
communicate asynchronously via external events, an architecture known as GALS 
(\emph{globally asynchronous--locally synchronous})~\cite{rp.gals}.

Concerning the described control-flow mechanisms, they heavily rely on 
\code{par/or} compositions, which cannot be precisely defined in asynchronous 
languages without tweaking processes with synchronization 
mechanisms~\cite{esterel.preemption}.

Finally, although \CEU provides some dataflow functionality, it is not intended 
for data-intensive applications.
For instance, the \emph{Functional Reactive Programming (FRP)} is a more 
expressive paradigm with this respect, supporting the dynamic creation of 
signals at runtime~\cite{frp.principles}.
We believe that dataflow and imperative reactivity are complementary, but the 
latter is more suitable for control-intensive embedded systems that must deal 
with low-level I/O and handle explicit state.
\end{comment}

\section{Conclusion}
\label{sec.conclusion}

\begin{comment}

In this work, we presented a formal description of the control aspects of the 
reactive programming language \CEU and discussed how to detect unsafe 
properties of programs at compile time.
\CEU is based on Esterel, but introduces the stacked behavior for internal 
events and the static analysis for shared-memory concurrency.

\CEU achieves a high degree of reliability for embedded systems, while also 
embraces practical aspects, such as support for lock-free concurrency, 
low-level access to the platform, and advanced control-flow mechanisms.

We consider that providing safe shared-memory concurrency is a fundamental 
design choice of \CEU, given that low-level I/O is indispensable in the context 
of embedded systems (e.g. interfacing with sensors and actuators).

Embedded systems are still predominantly developed in the ``bare metal'', 
regardless of existing alternatives, probably due to the flexibility and 
popularity of $C$.
We believe that \CEU is an attractive alternative, given its unrestricted 
access to $C$ and rich set of concurrent control primitives (e.g., parallel 
compositions and internal events).
\end{comment}

\begin{comment}
Currently, \CEU is not intended for use in other reactive scenarios, such as 
desktop applications and games:
besides the impracticability of the static analysis for larger applications, 
\CEU does not support the dynamic creation of trails, which is essential for 
virtualizing resources (e.g., graphical widgets, AI units, etc.).

most bug are in control aspects
spaghetti code
not on C calls

static
no virtualization of devices
1x1

Main contribution, stacked execution,
fundamental for the bounded execution (used on proof)
and for deriving control primitives

TODO: evaluation
- compared to handcrafted...

- Non-features
    - exponential analysis

the static analysis also detects the exact mem size
it is impossible to overflow timers/spawns
looks like dynamic but it is not
optionally, defensive (just count the number)

\CEU still has the limitation of not creating new lines of execution at runtime
They are all known in advance to properly calculate the amount of memory
also, dynamic would require a more complex static analysis
Note that dynamically allocating threads is not common in ES (chibi, what 
else?)

On the way to a more in-depth qualitative approach, we are currently teaching 
\CEU{} as an alternative to \nesc{} in a hands-on WSN course in a high-school.
The students successfully implemented a simple multi-hop communication protocol 
in \CEU.
Also, the same format is being employed in an undergraduate course, but still 
in an early stage.
We will compare the achievements of the students with both languages and use 
the results in our evaluation.

\end{comment}

%\newpage
\bibliographystyle{abbrv}
%\small{
\bibliography{other,my}
%}

\begin{comment}
The examples in Figure~\ref{lst:subs} show how this form of subroutines has 
some significant limitations.
We assume that the bodies represented as \code{...} do not contain \code{await} 
statements (but example~\ref{lst:subs}.c explicitly includes one).
Once again, the external event \code{A} represents the reaction that ``calls'' 
the functions.

The example~\ref{lst:subs}.a shows that a subroutine can be called only a 
single time during a reaction chain (further calls have no effect):
The first trail calls \code{f} in reaction to \code{A} and blocks.
After the subroutine returns, the first trail tries to call the function again.
However, given that $await$ statements must be previously awaiting to react to 
an $emit$, the second call is ignored.

The example~\ref{lst:subs}.b shows that subroutines are non-recursive 
(recursive calls have no effect):
The recursive call in the first trail fails, because the subroutine is not 
awaiting event \code{g} in the moment its body calls itself.
Remember that events are unbuffered in \CEU.
The recursive call fails, but the subsequent call in reaction to the next 
occurrence of \code{A} behaves normally.

The example~\ref{lst:subs}.c shows that subroutines can only run a single 
instance at a time (calls to running subroutines have no effect):
The first trail calls \code{h}, which awaits event \code{i} in its body.
The second reaction to \code{A} calls the subroutine again.
However, the subroutine did not terminate and the second call is missed.

\begin{figure}[t]
%\rule{8.5cm}{0.37pt}
{\small
\begin{minipage}[t]{0.25\linewidth}
\begin{lstlisting}
 input void A;
 event void f;
 par/or do
   loop do
     await f;
     ...
   end

 with
   await A;
   emit f;
   emit f;
 end


     (6.a)
  single call
\end{lstlisting}
\end{minipage}
%
\hspace{0.7cm}
%
\begin{minipage}[t]{0.25\linewidth}
\begin{lstlisting}
input void A;
event void g;
par/or do
  loop do
    await g;
    ...
    emit g;
  end
with
  await A;
  emit g;
  await A;
  emit g;
end

    (6.b)
non-recursive
\end{lstlisting}
\end{minipage}
\hspace{0.7cm}
%
\begin{minipage}[t]{0.25\linewidth}
\begin{lstlisting}
input void A;
event void h,i;
par/or do
  loop do
    await h;
    ...
    await i;
  end
with
  await A;
  emit h;
  await A;
  emit h;
end

    (6.c)
single instance
\end{lstlisting}
\end{minipage}
\caption{ Subroutines in \CEU.
\label{lst:subs}
}
}
\end{figure}
\end{comment}

\end{document}
